%% Generated by Sphinx.
\def\sphinxdocclass{report}
\documentclass[letterpaper,10pt,english]{sphinxmanual}
\ifdefined\pdfpxdimen
   \let\sphinxpxdimen\pdfpxdimen\else\newdimen\sphinxpxdimen
\fi \sphinxpxdimen=.75bp\relax

\PassOptionsToPackage{warn}{textcomp}
\usepackage[utf8]{inputenc}
\ifdefined\DeclareUnicodeCharacter
 \ifdefined\DeclareUnicodeCharacterAsOptional
  \DeclareUnicodeCharacter{"00A0}{\nobreakspace}
  \DeclareUnicodeCharacter{"2500}{\sphinxunichar{2500}}
  \DeclareUnicodeCharacter{"2502}{\sphinxunichar{2502}}
  \DeclareUnicodeCharacter{"2514}{\sphinxunichar{2514}}
  \DeclareUnicodeCharacter{"251C}{\sphinxunichar{251C}}
  \DeclareUnicodeCharacter{"2572}{\textbackslash}
 \else
  \DeclareUnicodeCharacter{00A0}{\nobreakspace}
  \DeclareUnicodeCharacter{2500}{\sphinxunichar{2500}}
  \DeclareUnicodeCharacter{2502}{\sphinxunichar{2502}}
  \DeclareUnicodeCharacter{2514}{\sphinxunichar{2514}}
  \DeclareUnicodeCharacter{251C}{\sphinxunichar{251C}}
  \DeclareUnicodeCharacter{2572}{\textbackslash}
 \fi
\fi
\usepackage{cmap}
\usepackage[T1]{fontenc}
\usepackage{amsmath,amssymb,amstext}
\usepackage{babel}
\usepackage{times}
\usepackage[Bjarne]{fncychap}
\usepackage{sphinx}

\usepackage{geometry}

% Include hyperref last.
\usepackage{hyperref}
% Fix anchor placement for figures with captions.
\usepackage{hypcap}% it must be loaded after hyperref.
% Set up styles of URL: it should be placed after hyperref.
\urlstyle{same}
\addto\captionsenglish{\renewcommand{\contentsname}{Contents:}}

\addto\captionsenglish{\renewcommand{\figurename}{Fig.}}
\addto\captionsenglish{\renewcommand{\tablename}{Table}}
\addto\captionsenglish{\renewcommand{\literalblockname}{Listing}}

\addto\captionsenglish{\renewcommand{\literalblockcontinuedname}{continued from previous page}}
\addto\captionsenglish{\renewcommand{\literalblockcontinuesname}{continues on next page}}

\addto\extrasenglish{\def\pageautorefname{page}}

\setcounter{tocdepth}{1}



\title{tcpb\_client Documentation}
\date{Oct 09, 2018}
\release{}
\author{Martinez Group}
\newcommand{\sphinxlogo}{\vbox{}}
\renewcommand{\releasename}{}
\makeindex

\begin{document}

\maketitle
\sphinxtableofcontents
\phantomsection\label{\detokenize{index::doc}}



\chapter{TeraChem Protocol Buffer (TCPB) Client}
\label{\detokenize{readme:welcome-to-tcpb-client-s-documentation}}\label{\detokenize{readme::doc}}\label{\detokenize{readme:terachem-protocol-buffer-tcpb-client}}
This repository is designed to facilitate the development a Python client for communicating with TeraChem.

This client uses C-style sockets for communication, and Protocol Buffers for a clean, well-defined way to serialize TeraChem input \& output.

For more information, see the \sphinxhref{https://bitbucket.org/mtzcloud/tcpb-python/wiki/Home}{Wiki}.


\section{Contact}
\label{\detokenize{readme:wiki}}\label{\detokenize{readme:contact}}\begin{itemize}
\item {} 
Stefan Seritan \textless{}\sphinxhref{mailto:sseritan@stanford.edu}{sseritan@stanford.edu}\textgreater{}

\end{itemize}


\chapter{tcpb}
\label{\detokenize{modules:tcpb}}\label{\detokenize{modules::doc}}

\section{tcpb package}
\label{\detokenize{tcpb::doc}}\label{\detokenize{tcpb:tcpb-package}}

\subsection{Submodules}
\label{\detokenize{tcpb:submodules}}

\subsection{tcpb.tcpb module}
\label{\detokenize{tcpb:tcpb-tcpb-module}}\label{\detokenize{tcpb:module-tcpb.tcpb}}\index{tcpb.tcpb (module)}
Simple Python socket client for communicating with TeraChem Protocol Buffer servers
\index{TCProtobufClient (class in tcpb.tcpb)}

\begin{fulllineitems}
\phantomsection\label{\detokenize{tcpb:tcpb.tcpb.TCProtobufClient}}\pysiglinewithargsret{\sphinxbfcode{\sphinxupquote{class }}\sphinxcode{\sphinxupquote{tcpb.tcpb.}}\sphinxbfcode{\sphinxupquote{TCProtobufClient}}}{\emph{host}, \emph{port}, \emph{debug=False}, \emph{trace=False}}{}
Bases: \sphinxhref{https://docs.python.org/3/library/functions.html\#object}{\sphinxcode{\sphinxupquote{object}}}

Connect and communicate with a TeraChem instance running in Protocol Buffer server mode
(i.e. TeraChem was started with the -s\textbar{}\textendash{}server flag)
\index{\_\_init\_\_() (tcpb.tcpb.TCProtobufClient method)}

\begin{fulllineitems}
\phantomsection\label{\detokenize{tcpb:tcpb.tcpb.TCProtobufClient.__init__}}\pysiglinewithargsret{\sphinxbfcode{\sphinxupquote{\_\_init\_\_}}}{\emph{host}, \emph{port}, \emph{debug=False}, \emph{trace=False}}{}
Initialize a TCProtobufClient object.
\begin{quote}\begin{description}
\item[{Parameters}] \leavevmode\begin{itemize}
\item {} 
\sphinxstyleliteralstrong{\sphinxupquote{host}} (\sphinxhref{https://docs.python.org/3/library/stdtypes.html\#str}{\sphinxstyleliteralemphasis{\sphinxupquote{str}}}) \textendash{} Hostname

\item {} 
\sphinxstyleliteralstrong{\sphinxupquote{port}} (\sphinxhref{https://docs.python.org/3/library/functions.html\#int}{\sphinxstyleliteralemphasis{\sphinxupquote{int}}}) \textendash{} Port number (must be above 1023)

\item {} 
\sphinxstyleliteralstrong{\sphinxupquote{debug}} (\sphinxhref{https://docs.python.org/3/library/functions.html\#bool}{\sphinxstyleliteralemphasis{\sphinxupquote{bool}}}) \textendash{} If True, assumes connections work (used for testing with no server)

\item {} 
\sphinxstyleliteralstrong{\sphinxupquote{trace}} (\sphinxhref{https://docs.python.org/3/library/functions.html\#bool}{\sphinxstyleliteralemphasis{\sphinxupquote{bool}}}) \textendash{} If True, packets are saved to .bin files (which can then be used for testing)

\end{itemize}

\end{description}\end{quote}

\end{fulllineitems}

\index{check\_job\_complete() (tcpb.tcpb.TCProtobufClient method)}

\begin{fulllineitems}
\phantomsection\label{\detokenize{tcpb:tcpb.tcpb.TCProtobufClient.check_job_complete}}\pysiglinewithargsret{\sphinxbfcode{\sphinxupquote{check\_job\_complete}}}{}{}
Pack and send a Status message to the TeraChem Protobuf server asynchronously.
This function expects a Status message back with either working or completed set.
Errors out if just busy message returned, implying the job we are checking was not submitted
or had some other issue
\begin{quote}\begin{description}
\item[{Returns}] \leavevmode
True if job is completed, False otherwise

\item[{Return type}] \leavevmode
\sphinxhref{https://docs.python.org/3/library/functions.html\#bool}{bool}

\end{description}\end{quote}

\end{fulllineitems}

\index{compute\_ci\_overlap() (tcpb.tcpb.TCProtobufClient method)}

\begin{fulllineitems}
\phantomsection\label{\detokenize{tcpb:tcpb.tcpb.TCProtobufClient.compute_ci_overlap}}\pysiglinewithargsret{\sphinxbfcode{\sphinxupquote{compute\_ci\_overlap}}}{\emph{geom=None}, \emph{geom2=None}, \emph{cvec1file=None}, \emph{cvec2file=None}, \emph{orb1afile=None}, \emph{orb1bfile=None}, \emph{orb2afile=None}, \emph{orb2bfile=None}, \emph{unitType='bohr'}, \emph{**kwargs}}{}
Compute wavefunction overlap given two different geometries, CI vectors, and orbitals,
using the same atom labels/charge/spin multiplicity as the previous calculation.

To run a closed shell calculation, only populate orb1afile/orb2afile, leaving orb1bfile/orb2bfile blank.
Currently, open-shell overlap calculations are not supported by TeraChem.
\begin{quote}\begin{description}
\item[{Parameters}] \leavevmode\begin{itemize}
\item {} 
\sphinxstyleliteralstrong{\sphinxupquote{geom}} \textendash{} Cartesian geometry of the first point

\item {} 
\sphinxstyleliteralstrong{\sphinxupquote{geom2}} \textendash{} Cartesian geometry of the second point

\item {} 
\sphinxstyleliteralstrong{\sphinxupquote{cvec1file}} \textendash{} Binary file of CI vector for first geometry (row-major, double64)

\item {} 
\sphinxstyleliteralstrong{\sphinxupquote{cvec2file}} \textendash{} Binary file of CI vector for second geometry (row-major, double64)

\item {} 
\sphinxstyleliteralstrong{\sphinxupquote{orb1afile}} \textendash{} Binary file of alpha MO coefficients for first geometry (row-major, double64)

\item {} 
\sphinxstyleliteralstrong{\sphinxupquote{orb1bfile}} \textendash{} Binary file of beta MO coefficients for first geometry (row-major, double64)

\item {} 
\sphinxstyleliteralstrong{\sphinxupquote{orb2afile}} \textendash{} Binary file of alpha MO coefficients for second geometry (row-major, double64)

\item {} 
\sphinxstyleliteralstrong{\sphinxupquote{orb2bfile}} \textendash{} Binary file of beta MO coefficients for second geometry (row-major, double64)

\item {} 
\sphinxstyleliteralstrong{\sphinxupquote{unitType}} \textendash{} Unit type key, as defined in the pb.Mol.UnitType enum (defaults to ‘bohr’)

\item {} 
\sphinxstyleliteralstrong{\sphinxupquote{**kwargs}} \textendash{} Additional TeraChem keywords, check \_process\_kwargs for behaviour

\end{itemize}

\item[{Returns}] \leavevmode
CI vector overlaps

\item[{Return type}] \leavevmode
(num\_states, num\_states) ndarray

\end{description}\end{quote}

\end{fulllineitems}

\index{compute\_coupling() (tcpb.tcpb.TCProtobufClient method)}

\begin{fulllineitems}
\phantomsection\label{\detokenize{tcpb:tcpb.tcpb.TCProtobufClient.compute_coupling}}\pysiglinewithargsret{\sphinxbfcode{\sphinxupquote{compute\_coupling}}}{\emph{geom=None}, \emph{unitType='bohr'}, \emph{**kwargs}}{}
Compute nonadiabatic coupling of a new geometry, but with the same atoms labels/charge/spin
multiplicity and wave function format as the previous calculation.
\begin{quote}\begin{description}
\item[{Parameters}] \leavevmode\begin{itemize}
\item {} 
\sphinxstyleliteralstrong{\sphinxupquote{geom}} \textendash{} Cartesian geometry of the new point

\item {} 
\sphinxstyleliteralstrong{\sphinxupquote{unitType}} \textendash{} Unit type key, as defined in the pb.Mol.UnitType enum (defaults to ‘bohr’)

\item {} 
\sphinxstyleliteralstrong{\sphinxupquote{**kwargs}} \textendash{} Additional TeraChem keywords, check \_process\_kwargs for behaviour

\end{itemize}

\item[{Returns}] \leavevmode
Nonadiabatic coupling vector

\item[{Return type}] \leavevmode
(num\_atoms, 3) ndarray

\end{description}\end{quote}

\end{fulllineitems}

\index{compute\_energy() (tcpb.tcpb.TCProtobufClient method)}

\begin{fulllineitems}
\phantomsection\label{\detokenize{tcpb:tcpb.tcpb.TCProtobufClient.compute_energy}}\pysiglinewithargsret{\sphinxbfcode{\sphinxupquote{compute\_energy}}}{\emph{geom=None}, \emph{unitType='bohr'}, \emph{**kwargs}}{}
Compute energy of a new geometry, but with the same atom labels/charge/spin
multiplicity and wave function format as the previous calculation.
\begin{quote}\begin{description}
\item[{Parameters}] \leavevmode\begin{itemize}
\item {} 
\sphinxstyleliteralstrong{\sphinxupquote{geom}} \textendash{} Cartesian geometry of the new point

\item {} 
\sphinxstyleliteralstrong{\sphinxupquote{unitType}} \textendash{} Unit type key, as defined in the pb.Mol.UnitType enum (defaults to ‘bohr’)

\item {} 
\sphinxstyleliteralstrong{\sphinxupquote{**kwargs}} \textendash{} Additional TeraChem keywords, check \_process\_kwargs for behaviour

\end{itemize}

\item[{Returns}] \leavevmode
Energy

\item[{Return type}] \leavevmode
\sphinxhref{https://docs.python.org/3/library/functions.html\#float}{float}

\end{description}\end{quote}

\end{fulllineitems}

\index{compute\_forces() (tcpb.tcpb.TCProtobufClient method)}

\begin{fulllineitems}
\phantomsection\label{\detokenize{tcpb:tcpb.tcpb.TCProtobufClient.compute_forces}}\pysiglinewithargsret{\sphinxbfcode{\sphinxupquote{compute\_forces}}}{\emph{geom=None}, \emph{unitType='bohr'}, \emph{**kwargs}}{}
Compute forces of a new geometry, but with the same atoms labels/charge/spin
multiplicity and wave function format as the previous calculation.
\begin{quote}\begin{description}
\item[{Parameters}] \leavevmode\begin{itemize}
\item {} 
\sphinxstyleliteralstrong{\sphinxupquote{geom}} \textendash{} Cartesian geometry of the new point

\item {} 
\sphinxstyleliteralstrong{\sphinxupquote{unitType}} \textendash{} Unit type key, as defined in the pb.Mol.UnitType enum (defaults to ‘bohr’)

\item {} 
\sphinxstyleliteralstrong{\sphinxupquote{**kwargs}} \textendash{} Additional TeraChem keywords, check \_process\_kwargs for behaviour

\end{itemize}

\item[{Returns}] \leavevmode
Tuple of (energy, forces), which is really (energy, -gradient)

\item[{Return type}] \leavevmode
\sphinxhref{https://docs.python.org/3/library/stdtypes.html\#tuple}{tuple}

\end{description}\end{quote}

\end{fulllineitems}

\index{compute\_gradient() (tcpb.tcpb.TCProtobufClient method)}

\begin{fulllineitems}
\phantomsection\label{\detokenize{tcpb:tcpb.tcpb.TCProtobufClient.compute_gradient}}\pysiglinewithargsret{\sphinxbfcode{\sphinxupquote{compute\_gradient}}}{\emph{geom=None}, \emph{unitType='bohr'}, \emph{**kwargs}}{}
Compute gradient of a new geometry, but with the same atom labels/charge/spin
multiplicity and wave function format as the previous calculation.
\begin{quote}\begin{description}
\item[{Parameters}] \leavevmode\begin{itemize}
\item {} 
\sphinxstyleliteralstrong{\sphinxupquote{geom}} \textendash{} Cartesian geometry of the new point

\item {} 
\sphinxstyleliteralstrong{\sphinxupquote{unitType}} \textendash{} Unit type key, as defined in the pb.Mol.UnitType enum (defaults to ‘bohr’)

\item {} 
\sphinxstyleliteralstrong{\sphinxupquote{**kwargs}} \textendash{} Additional TeraChem keywords, check \_process\_kwargs for behaviour

\end{itemize}

\item[{Returns}] \leavevmode
Tuple of (energy, gradient)

\item[{Return type}] \leavevmode
\sphinxhref{https://docs.python.org/3/library/stdtypes.html\#tuple}{tuple}

\end{description}\end{quote}

\end{fulllineitems}

\index{compute\_job\_sync() (tcpb.tcpb.TCProtobufClient method)}

\begin{fulllineitems}
\phantomsection\label{\detokenize{tcpb:tcpb.tcpb.TCProtobufClient.compute_job_sync}}\pysiglinewithargsret{\sphinxbfcode{\sphinxupquote{compute\_job\_sync}}}{\emph{jobType='energy'}, \emph{geom=None}, \emph{unitType='bohr'}, \emph{**kwargs}}{}
Wrapper for send\_job\_async() and recv\_job\_async(), using check\_job\_complete() to poll the server.
\begin{quote}\begin{description}
\item[{Parameters}] \leavevmode\begin{itemize}
\item {} 
\sphinxstyleliteralstrong{\sphinxupquote{jobType}} \textendash{} Job type key, as defined in the pb.JobInput.RunType enum (defaults to ‘energy’)

\item {} 
\sphinxstyleliteralstrong{\sphinxupquote{geom}} \textendash{} Cartesian geometry of the new point

\item {} 
\sphinxstyleliteralstrong{\sphinxupquote{unitType}} \textendash{} Unit type key, as defined in the pb.Mol.UnitType enum (defaults to ‘bohr’)

\item {} 
\sphinxstyleliteralstrong{\sphinxupquote{**kwargs}} \textendash{} Additional TeraChem keywords, check \_process\_kwargs for behaviour

\end{itemize}

\item[{Returns}] \leavevmode
Results mirroring recv\_job\_async

\item[{Return type}] \leavevmode
\sphinxhref{https://docs.python.org/3/library/stdtypes.html\#dict}{dict}

\end{description}\end{quote}

\end{fulllineitems}

\index{connect() (tcpb.tcpb.TCProtobufClient method)}

\begin{fulllineitems}
\phantomsection\label{\detokenize{tcpb:tcpb.tcpb.TCProtobufClient.connect}}\pysiglinewithargsret{\sphinxbfcode{\sphinxupquote{connect}}}{}{}
Connect to the TeraChem Protobuf server

\end{fulllineitems}

\index{disconnect() (tcpb.tcpb.TCProtobufClient method)}

\begin{fulllineitems}
\phantomsection\label{\detokenize{tcpb:tcpb.tcpb.TCProtobufClient.disconnect}}\pysiglinewithargsret{\sphinxbfcode{\sphinxupquote{disconnect}}}{}{}
Disconnect from the TeraChem Protobuf server

\end{fulllineitems}

\index{is\_available() (tcpb.tcpb.TCProtobufClient method)}

\begin{fulllineitems}
\phantomsection\label{\detokenize{tcpb:tcpb.tcpb.TCProtobufClient.is_available}}\pysiglinewithargsret{\sphinxbfcode{\sphinxupquote{is\_available}}}{}{}
Asks the TeraChem Protobuf server whether it is available or busy through the Status protobuf message.
Note that this does not reserve the server, and the status could change after this function is called.
\begin{quote}\begin{description}
\item[{Returns}] \leavevmode
True if the TeraChem PB server is currently available (no running job)

\item[{Return type}] \leavevmode
\sphinxhref{https://docs.python.org/3/library/functions.html\#bool}{bool}

\end{description}\end{quote}

\end{fulllineitems}

\index{recv\_job\_async() (tcpb.tcpb.TCProtobufClient method)}

\begin{fulllineitems}
\phantomsection\label{\detokenize{tcpb:tcpb.tcpb.TCProtobufClient.recv_job_async}}\pysiglinewithargsret{\sphinxbfcode{\sphinxupquote{recv\_job\_async}}}{}{}
Recv and unpack a JobOutput message from the TeraChem Protobuf server asynchronously.
This function expects the job to be ready (i.e. check\_job\_complete() returned true),
so will error out on timeout.

Creates a results dictionary that mirrors the JobOutput message, using NumPy arrays when appropriate.
Results are also saved in the prev\_results class member.
An inclusive list of the results members (with types):
\begin{itemize}
\item {} 
atoms:              Flat \# of atoms NumPy array of 2-character strings

\item {} 
geom:               \# of atoms by 3 NumPy array of doubles

\item {} 
energy:             Either empty, single energy, or flat \# of cas\_energy\_labels of NumPy array of doubles

\item {} 
charges:            Flat \# of atoms NumPy array of doubles

\item {} 
spins:              Flat \# of atoms NumPy array of doubles

\item {} 
dipole\_moment:      Single element

\item {} 
dipole\_vector:      Flat 3-element NumPy array of doubles

\item {} 
job\_dir:            String

\item {} 
job\_scr\_dir:        String

\item {} 
server\_job\_id:      Int

\item {} 
orbfile:            String (if restricted is True, otherwise not included)

\item {} 
orbfile\_a:          String (if restricted is False, otherwise not included)

\item {} 
orbfile\_b:          String (if restricted is False, otherwise not included)

\item {} 
orb\_energies:       Flat \# of orbitals NumPy array of doubles (if restricted is True, otherwise not included)

\item {} 
orb\_occupations:    Flat \# of orbitals NumPy array of doubles (if restricted is True, otherwise not included)

\item {} 
orb\_energies\_a:     Flat \# of orbitals NumPy array of doubles (if restricted is False, otherwise not included)

\item {} 
orb\_occupations\_a:  Flat \# of orbitals NumPy array of doubles (if restricted is False, otherwise not included)

\item {} 
orb\_energies\_b:     Flat \# of orbitals NumPy array of doubles (if restricted is False, otherwise not included)

\item {} 
orb\_occupations\_b:  Flat \# of orbitals NumPy array of doubles (if restricted is False, otherwise not included)

\end{itemize}

Additional (optional) members of results:
\begin{itemize}
\item {} 
gradient:           \# of atoms by 3 NumPy array of doubles (available for ‘gradient’ job)

\item {} 
nacme:              \# of atoms by 3 NumPy array of doubles (available for ‘coupling’ job)

\item {} 
transition\_dipole:  Flat 3-element NumPy array of doubles (available for ‘coupling’ job)

\item {} 
cas\_energy\_labels:  List of tuples of (state, multiplicity) corresponding to the energy list

\item {} 
bond\_order:         \# of atoms by \# of atoms NumPy array of doubles

\item {} 
ci\_overlap:         ci\_overlap\_size by ci\_overlap\_size NumPy array of doubles (available for ‘ci\_vec\_overlap’ job)

\end{itemize}
\begin{quote}\begin{description}
\item[{Returns}] \leavevmode
Results as described above

\item[{Return type}] \leavevmode
\sphinxhref{https://docs.python.org/3/library/stdtypes.html\#dict}{dict}

\end{description}\end{quote}

\end{fulllineitems}

\index{send\_job\_async() (tcpb.tcpb.TCProtobufClient method)}

\begin{fulllineitems}
\phantomsection\label{\detokenize{tcpb:tcpb.tcpb.TCProtobufClient.send_job_async}}\pysiglinewithargsret{\sphinxbfcode{\sphinxupquote{send\_job\_async}}}{\emph{jobType='energy'}, \emph{geom=None}, \emph{unitType='bohr'}, \emph{**kwargs}}{}
Pack and send the current JobInput to the TeraChem Protobuf server asynchronously.
This function expects a Status message back that either tells us whether the job was accepted.
\begin{quote}\begin{description}
\item[{Parameters}] \leavevmode\begin{itemize}
\item {} 
\sphinxstyleliteralstrong{\sphinxupquote{jobType}} \textendash{} Job type key, as defined in the pb.JobInput.RunType enum (defaults to “energy”)

\item {} 
\sphinxstyleliteralstrong{\sphinxupquote{geom}} \textendash{} Cartesian geometry of the new point

\item {} 
\sphinxstyleliteralstrong{\sphinxupquote{unitType}} \textendash{} Unit type key, as defined in the pb.Mol.UnitType enum (defaults to “bohr”)

\item {} 
\sphinxstyleliteralstrong{\sphinxupquote{**kwargs}} \textendash{} Additional TeraChem keywords, check \_process\_kwargs for behaviour

\end{itemize}

\item[{Returns}] \leavevmode
True on job acceptance, False on server busy, and errors out if communication fails

\item[{Return type}] \leavevmode
\sphinxhref{https://docs.python.org/3/library/functions.html\#bool}{bool}

\end{description}\end{quote}

\end{fulllineitems}

\index{update\_address() (tcpb.tcpb.TCProtobufClient method)}

\begin{fulllineitems}
\phantomsection\label{\detokenize{tcpb:tcpb.tcpb.TCProtobufClient.update_address}}\pysiglinewithargsret{\sphinxbfcode{\sphinxupquote{update\_address}}}{\emph{host}, \emph{port}}{}
Update the host and port of a TCProtobufClient object.
Note that you will have to call disconnect() and connect() before and after this
yourself to actually connect to the new server.
\begin{quote}\begin{description}
\item[{Parameters}] \leavevmode\begin{itemize}
\item {} 
\sphinxstyleliteralstrong{\sphinxupquote{host}} (\sphinxhref{https://docs.python.org/3/library/stdtypes.html\#str}{\sphinxstyleliteralemphasis{\sphinxupquote{str}}}) \textendash{} Hostname

\item {} 
\sphinxstyleliteralstrong{\sphinxupquote{port}} (\sphinxhref{https://docs.python.org/3/library/functions.html\#int}{\sphinxstyleliteralemphasis{\sphinxupquote{int}}}) \textendash{} Port number (must be above 1023)

\end{itemize}

\end{description}\end{quote}

\end{fulllineitems}


\end{fulllineitems}



\subsection{Module contents}
\label{\detokenize{tcpb:module-tcpb}}\label{\detokenize{tcpb:module-contents}}\index{tcpb (module)}

\chapter{Indices and tables}
\label{\detokenize{index:indices-and-tables}}\begin{itemize}
\item {} 
\DUrole{xref,std,std-ref}{genindex}

\item {} 
\DUrole{xref,std,std-ref}{modindex}

\item {} 
\DUrole{xref,std,std-ref}{search}

\end{itemize}


\renewcommand{\indexname}{Python Module Index}
\begin{sphinxtheindex}
\def\bigletter#1{{\Large\sffamily#1}\nopagebreak\vspace{1mm}}
\bigletter{t}
\item {\sphinxstyleindexentry{tcpb}}\sphinxstyleindexpageref{tcpb:\detokenize{module-tcpb}}
\item {\sphinxstyleindexentry{tcpb.tcpb}}\sphinxstyleindexpageref{tcpb:\detokenize{module-tcpb.tcpb}}
\end{sphinxtheindex}

\renewcommand{\indexname}{Index}
\printindex
\end{document}